%\renewcommand{\thechapter}{8}

\chapter{Summary and discussion}

In this thesis, we have presented fundamental results pertaining to passive scalar
turbulence in kinetic systems---we analytically derived the fluctuation-dissipation
relations for a kinetic scalar (\chapref{chap:phmixlin}), and we showed numerical results
which shed light on how linear phase mixing for a kinetic passive scalar is modified due
to nonlinear advection (chapters \ref{chap:pp0} and \ref{chap:phmixnl}). In particular, in
\chapref{chap:phmixnl}, we identified the turbulent analog of the plasma echo, and
demonstrated,
with the aid of a simple model, that phase mixing may be siginificantly suppressed due to the
echo in collisionless systems. We developed a new code, \Gand\ (\chapref{chap:gandalf}) to simulate the kinetic
reduced MHD equations \cite{tome}, which describe the Alfv\'{e}nic and compressive
components of the turbulent cascade in the solar wind at scales larger than the ion Larmor
radius. In \chapref{chap:slowmodes}, we addressed two key questions regarding the
compressive cascade in the inertial range using numerical simulations ---
\begin{inparaenum}
\item Do the compressive fluctuations have a parallel cascade?
\item Why are the density and field strength fluctuations undamped at kinetic scales?
\end{inparaenum}
We found that the compressive perturbations do indeed have a parallel cascade, and have
the same perpendicular and parallel power law spectra as the Alfv\'{e}nic 
fluctuations. We showed, by diagnosing the
flux in Hermite space, that despite the parallel cascade compressive fluctuations remain
undamped due to the stochastic plasma echo. 

Even in the absence of a parallel cascade for the slow modes, the power law spectra can be
explained using the results from \chapref{chap:pp0}. In this limit, the slow modes cascade
to small perpendicular spatial scales before they can phase mix, resulting in a fluid-like
turbulent cascade, hence, exhibit power law spectra. 

We would like to investigate the cascade of compressive fluctuations further, in
particular, to understand how the parallel cascade comes about.
After developing solid understanding of this problem, we hope to study how the compressive cascade, in
particular the plasma echo, is dependent on parameters like the plasma beta, the ion
charge, and the ion to
electron temperature ratio. Another direction forward would be to add more physics to
\Gand. This can be done in two possible ways. Firstly, by implementing non-Maxwellian
distribution functions---this would enable 
further numerical investigations into solar wind turbulence near the marginal stability
boundaries for firehose and mirror instabilities. Secondly, making \Gand\ fully gyrokinetic
by including finite Larmor radius effects. The work in this thesis was restricted to
scales larger than the ion Larmor radius, where linear phase mixing is the only 
phase mixing process available to the system.
With a gyrokinetic code, the ideas developed here could be generalized to include finite
Larmor radius effects, specifically, to investigate the role nonlinear phase mixing
\cite{tatsuno09} plays in the turbulent cascade at sub-Larmor scales.

The analytical and numerical framework developed as a part of this thesis fits in a larger
program to understand the properties of turbulence in weakly collisional magnetized
plasmas like the solar wind, in particular, to study the dissipation mechanisms favored by
the system, and to learn how the dissipated energy is partitioned between different
species of the plasma.
