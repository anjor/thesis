%\renewcommand{\thechapter}{2}
\chapter{Mathematical framework: Kinetic reduced MHD}
\label{app:eq}
    
    
    \section{Introduction}

    A kinetic plasma is described by the distribution function 
    $f_s(t, \mb{r},\mb{v})$---the probability of finding a particle of species $s$ (ions or electrons) at
    position $\mb{r}$ with velocity $\mb{v}$ at time $t$. The distribution function
    evolves according to the Boltzmann equation:
    \beq
        \pd{f_s}{t} + \mb{v}\cdot\nabla f_s + \frac{q_s}{m_s} \lt(\mb{E} +
        \frac{\mb{v}\times\mb{B}}{c}\rt)\cdot\pd{f_s}{\mb{v}} =
        \lt(\pd{f_s}{t}\rt)_{\text{coll}},
        \label{eqs:eq:boltzmann}
    \eeq
    where $q_s$ and $m_s$ are the species' charge and mass, $c$ is the speed of light; the
    right hand side is the (quadratic in $f$) Landau collision operator. The electric and magnetic fields,
    $\mb{E}$ and $\mb{B}$, are calculated using Maxwell's equations:
    \beq
        \nabla \times \mb{B} = \frac{4 \pi}{c} \mb{j} + \frac{1}{c}\pd{\mb{E}}{t},
        \label{eqs:eq:ampere}
    \eeq
    \beq
        \nabla \times \mb{E} = - \frac{1}{c} \pd{\mb{B}}{t}, \label{eqs:eq:faraday}
    \eeq
    \beq
        \nabla\cdot\mb{E} = 4 \pi \rho,\label{eqs:eq:poisson}
    \eeq
    \beq
        \nabla\cdot \mb{B} = 0,\label{eqs:eq:divb}
    \eeq
    \beq
        \rho = \sum_s q_s \int d^3 \mb{v} f_s, \quad \mb{j} = \sum_s q_s \int
        d^3\mb{v}\mb{v} f_s,
    \eeq
    where $\rho$ and $\mb{j}$ are the charge and current densities. 

    Although a complete description, the Boltzmann-Maxwell set of
    \eqsdash{eqs:eq:boltzmann}{eqs:eq:divb} is computationally prohibitively expensive. A
    more tractable,
    reduced set of equations can be derived by limiting to a description of plasmas with a strong mean magnetic
    field. It is assumed that the turbulent fluctuations in such a plasma are 
    \begin{inparaenum}[(i)] 
    \item spatially anisotropic with respect to the mean field, 
    \item have frequencies that are smaller than the ion cyclotron frequency, 
    \item and are small in amplitude in comparison with the equilibrium quantities. 
    \end{inparaenum}
    The dimensionality of the phase space is then reduced from six to five by averaging
    over the fast cyclotron motion of the particles. This set of equations is known as
    gyrokinetics, which though simpler, is still a fully kinetic description.

    Depending on the research problems in mind, even simpler, hybrid models can be derived
    by making further approximations. All the results in this thesis pertain 
    to the behavior of weakly collisional plasmas at scales larger than
    the ion gyro-radius: the so called ``inertial range". By expanding the gyrokinetic
    equation in the smallness of the ion gyro-radius, one can derive kinetic reduced
    magnetohydrodynamics, an asymptotically valid description
    of these systems in the inertial range. All the work in this thesis is done in the limit
    where KRMHD is true. In the next two sections we describe the gyrokinetic and the
    KRMHD set of equations.
    

    \section{Gyrokinetics}
    \label{eqs:sec:gk}

%    \begin{enumerate}
%        \item History of gyrokinetics
%        \item Gyrokinetic ordering
%        \item Full set of equations
%    \end{enumerate}


        \subsection{Introduction}
    Linear \cite{rutherford68, taylor68, catto78, antonsen80, catto81} and nonlinear gyrokinetics
    \cite{frieman82, dubin83, lee83, lee87, hahm88, howes06, tome, abel13} has been
    used in studying magnetized plasmas for over four decades. 
    Historically, gyrokinetics has been a popular
    choice to study turbulence and transport generated by micro-instabilities in fusion
    plasmas (\cite{dimits96, dorland00, jenko00, rogers00, jenko01, jenko01ppcf, jenko02,
    candy04, parker04} are a handful of examples). 
    In the past ten years, however, there has been substantial work studying the relevance
    and utility of gyrokinetics for space and astrophysical plasmas \cite{howes06,
    howes07, howes08pop, schekochihin08, tome, numata10, howes11prl, tenbarge12, tenbarge13, tenbarge13a,
    tenbarge14} . Traditionally, these
    plasmas have been described using magnetohydrodynamics. However,
    there are many examples of astrophysical plasmas where small-scale perturbations have
    wavelengths smaller than the ion mean free path, and therefore require a kinetic
    description. MHD turbulence has a natural propensity to drive the system towards
    increasing anisotropy as energy is cascaded to small scales \cite{galtier00,
    schekochihin12}. The intrinsic anisotropic
    nature of the MHD turbulent cascade also implies that the frequencies of these small-scale
    fluctuations remain far below the ion cyclotron frequency (this is so because the
    frequencies of the turbulence are proportional to the parallel wavenumbers). Hence, such plasmas are well
    described by gyrokinetics.

    \subsection{Equations}
    First separate the distribution function and the fields into equilibrium and
    fluctuating parts (the $\delta f$ approximation):
    \beq
        f_s = F_{0s} + \delta f_s, \,\, \mb{B} = \mb{B}_0 + \delta\mb{B}, \,\, \mb{E} =
        \delta \mb{E},
    \eeq
    where $F_{0s}$ is the equilibrium distribution function, which to zeroth order is a
    Maxwellian:
    \beq
      F_{0s} = \frac{n_{0s}}{\lt(\pi v_{ths}^2\rt)^{3/2}} \exp
      \lt(-\frac{v^2}{v_{ths}^2}\rt), \quad v_{ths} = \sqrt{\frac{2 T_{0s}}{m_s}},
    \eeq
    $n_{0s}$ and $T_{0s}$ are the density and temperature of species $s$. We further
    assume that the equilibrium is homogeneous, i.e., there are no gradients of the
    equilibrium density and temperature. The background magnetic field $\mb{B_0}$ is 
    assumed to be a straight, uniform magnetic field:
    \beq
        \mb{B}_0 = B_0 \hat{\mb{z}}.
    \eeq
    The two homogeneous Maxwell's
    \eqsand{eqs:eq:faraday}{eqs:eq:divb} can be solved by expressing the fields in terms
    of potentials,
    \beq
        \delta \mb{E} = - \nabla \phi - \frac{1}{c}\pd{\Apar}{t}, \quad \delta \mb{B} = \nabla \times
        \mb{A},
    \eeq
    $\phi$ and $\mb{A}$ are the electrostatic and magnetic vector potentials respectively;
    we also choose the Coulomb gauge $\nabla \cdot \mb{A} = 0$.
    
    The gyrokinetic approximation is formalized by the following ordering
    assumptions:
    \beq
        \frac{\delta f_s}{F_{0s}} \sim \frac{\delta B}{B_0} \sim \frac{\delta
        E}{(v_{ths}/c)B_0} \sim \frac{\kpar}{k_\perp} \sim \frac{\omega}{\Omega_i} \sim \epsilon \ll 1,
    \eeq
    where $\kpar$ and $k_\perp$ are the spatial wavenumbers along and across the magnetic
    field, $\omega$ is the typical frequency of the fluctuations and $\Omega_i$ is the ion
    cyclotron frequency.

    The perturbed distribution function can be further split into two parts,
    \beq
        \delta f_s = -\frac{q_s \phi(t, \mb{r})}{T_{0s}} F_{0s} + h_s(t, \mb{R}_s,v_\perp,
        \vpar),
    \eeq
    where the first term is the Boltzmann response. The second term is the
    distribution function of the centers of the particle gyro-orbits. Note that the gyrocenter distribution
    function is evaluated at the guiding center position $\mb{R}_s$ and not at the particle
    position $\mb{r}$,
    \beq
        \mb{R}_s = \mb{r} + \frac{\mb{v}_\perp \times \hat{\mb{z}}}{\Omega_s},
    \eeq
    and is a function of the velocity space variables $v_\perp$ and $\vpar$ \footnote{This
    choice of velocity space co-ordinates is convenient for homogeneous plasmas. For
    inhomogeneous plasmas, the conserved quantities energy and magnetic moment make better
    co-ordinates \cite{frieman82}.}.
   % The choice of velocity space variables $v_\perp$ and $\vpar$ is convenient for
   % homogeneous plasmas; for inhomogeneous plasmas, on the other hand, energy and magnetic
   % moment are better velocity space co-ordinates. 
    The function $h_s$ satisfies the
    gyrokinetic equation:
    \beq
        \pd{h_s}{t} + \vpar \pd{h_s}{z} + \frac{c}{B_0}\lt\{\langle\chi\rangle_{\mb{R_s}},h_s\rt\}  =
        \frac{q_s F_{0s}}{T_{0s}} \pd{\langle\chi\rangle_{\mb{R_s}}}{t} +
        \lt(\pd{h_s}{t}\rt)_{\text{coll}}, \label{eqs:eq:gk}
    \eeq
    where $\chi$ is the gyrokinetic potential,
    \beq
        \chi = \phi - \frac{\vpar \Apar}{c} - \frac{\mb{v}_\perp\cdot \mb{A}_\perp}{c}.
    \eeq
    The Poisson bracket is defined as,
    \beq
        \lt\{ P, Q\rt\} = \hat{\mb{z}}\cdot \lt(\pd{P}{\mb{R}_s} \times
        \pd{Q}{\mb{R}_s}\rt).
    \eeq
    The angle brackets in \eqref{eqs:eq:gk} denote an average over the Larmor motion of the
    particle at a fixed guiding center position:
    \beq
        \langle \chi\lt(t, \mb{r},\vpar, \mb{v}_\perp\rt) \rangle_{\mb{R}_s} =
        \frac{1}{2\pi}\int_0^{2 \pi} d \theta \, \chi \lt(t, \mb{R}_s -
        \frac{\mb{v_\perp}\times\hat{\mb{z}}}{\Omega_s}, v_\perp, \vpar \rt),
        \label{eqs:eq:ringaveR}
    \eeq
    where $\theta$ is the angular velocity-space co-ordinate in a cyclindrical
    co-ordinate system:
    \beq
        \mb{v} = \vpar \hat{\mb{z}} + v_\perp \lt(\cos \theta \hat{\mb{x}} + \sin
        \theta\hat{\mb{y}}\rt).
    \eeq
    Observe that the ring average in \eqref{eqs:eq:ringaveR} is evaluated at constant
    guiding center position, but the gyrokinetic potential is a function of the particle
    position. Another thing to notice is that the ring average, as well as the guiding
    center position $\mb{R}_s$ depends on the particle species index $s$.
    
    The electromagnetic fields are calculated consistently from $h_s$ using Maxwell's
    equations. In the non-relativistic limit, Poisson's \eqref{eqs:eq:poisson} turns into
    a quasineutrality condition,
    \beq
       0 = \sum_s q_s \delta n_s = \sum_s q_s \lt[\frac{-q_s \phi}{T_{0s}} n_{0s} + \int
       d^3 \mb{v} \langle h_s \rangle_{\mb{r}}\rt]; \label{eqs:eq:phi}
    \eeq
    the parallel and perpendicular components of the Ampere's law take the following forms:
    \beq
       \nabla_\perp^2 \Apar = -\frac{4\pi}{c} j_\parallel = -\frac{4\pi}{c} \sum_s q_s
       \int d^3 \mb{v} \vpar \langle h_s \rangle_{\mb{r}}, \label{eqs:eq:apar}
    \eeq
    \beq
       \nabla_\perp^2 \dBpar = - \frac{4\pi}{c}
       \hat{\mb{z}}\cdot\lt(\nabla_\perp\times\mb{j_\perp}\rt) = 
       -\frac{4\pi}{c} \hat{\mb{z}}\cdot\lt[\nabla_\perp \times \sum_s q_s \int
       d^3\mb{v}\langle\mb{v}_\perp h_s\rangle_{\mb{r}}\rt], \label{eqs:eq:bpar}
    \eeq
    where $\delta \Bpar =
    \hat{\mb{z}}\cdot\lt(\nabla_\perp\times\mb{A}_\perp\rt)$ is the field strength
    fluctuation. Notice that since the
    electromagnetic field variables $\phi$, $\Apar$ and $\dBpar$ are functions of the
    particle position and not the guiding center position, the charge and current
    densities are
    calculated by gyroaveraging the guiding center distribution at fixed $\mb{r}$ (a dual
    operation to the one in \eqref{eqs:eq:ringaveR}),
    \beq
        \langle h_s\lt(t, \mb{R}_s,\vpar, \mb{v}_\perp\rt) \rangle_{\mb{r}} =
        \frac{1}{2\pi}\int_0^{2 \pi} d \theta \, \chi \lt(t, \mb{r} +
        \frac{\mb{v_\perp}\times\hat{\mb{z}}}{\Omega_s}, v_\perp, \vpar \rt).
        \label{eqs:eq:ringaver}
    \eeq

    Gyrokinetic \eqref{eqs:eq:gk} for ions and electrons, along with the field
    \eqsdash{eqs:eq:phi}{eqs:eq:bpar} form a complete, self-consistent set of equations.

    \subsection{Conserved quantity}
    \label{eqs:sec:gk:consqty}

    In absence of collisions, the gyrokinetic system of equations conserves the following quantity, which is the
    gyrokinetic version of the free energy:
    \beq
        W = \int d^3\mb{r}\lt[\sum_s \lt(\int d^3 \mb{v} \frac{T_{0s} \langle
        h_s^2\rangle_{\mb{r}}}{2 F_{0s}} - \frac{q_s^2 \phi^2 n_{0s}}{T_{0s}}\rt) +
        \frac{\lt|\delta \mb{B}\rt|^2}{8\pi}\rt] = 
        \int d^3 \mb{r}\lt(\sum_s \int d^3 \mb{v} \frac{T_{0s}\delta f_s^2}{2 F_{0s}} +
        \frac{\lt|\delta \mb{B}\rt|^2}{8 \pi} \rt).
    \eeq
    $W$ is the quantity that is cascaded in the phase-space in gyrokinetic turbulence,
    and is eventually destroyed by collisions, which generates entropy and heats the
    plasma.


%    \section{$k_\perp \rho_e \ll 1$: Isothermal electrons}
%
%    \begin{enumerate}
%        \item $k \rho_e$ small, how this relates to mass ratio expansion
%        \item Physical explanation for each equation: reconnection disallowed
%    \end{enumerate}
%
%    \beq
%        \frac{1}{c}\pd{\Apar}{t} + \hat{\mb{b}}\cdot\nabla\phi =
%        \hat{\mb{b}}\cdot\nabla\frac{T_{0e}}{e} \frac{\delta n_e}{n_{0e}},
%        \label{eqs:ief:epar}
%    \eeq
%    \beq
%        \od{}{t}\lt(\frac{\delta n_e}{n_{0e}} - \frac{\dBpar}{B_0}\rt) + \hat{\mb{b}}\cdot
%        \nabla u_{\parallel,e} = -\frac{c T_{0e}}{e B_0} \lt\{\frac{\delta n_e}{n_{0e}},
%        \frac{\dBpar}{B_0}\rt\}, \label{eqs:ief:vort}
%    \eeq
%    \beq
%        \frac{\delta n_e}{n_{0e}} = - \frac{Ze \phi}{T_{0i}} + \frac{1}{n_{0i}} \int
%        d^3\mb{v} \langle h_i\rangle_{\mb{r}}, \label{eqs:ief:dne}
%    \eeq
%    \beq
%        u_{\parallel,e} = \frac{c}{4 \pi e n_{0e}} \nabla_\perp^2 \Apar + \int d^3\mb{v}
%        \vpar \langle h_i \rangle_{\mb{r}},\label{eqs:ief:upar}
%    \eeq
%    \beq
%        \frac{\dBpar}{B_0} =
%        \frac{\beta_i}{2}\lt\{\lt(1+\frac{Z}{\tau}\rt)\frac{Ze\phi}{T_{0i}} - 
%        \sum_{\mb{k}} e^{i\mb{k}\cdot\mb{r}} \frac{1}{n_{0i}}\lt[\frac{Z}{\tau}J_0(a_i) +
%        \frac{2 v_\perp^2}{v_{thi}^2}\frac{J_1(a_i)}{a_i}\rt] h_{i\mb{k}}\rt\},
%        \label{eqs:ief:dbpar}
%    \eeq
%    \beq
%        \pd{h_i}{t} + \vpar \pd{h_i}{z} + \frac{c}{B_0}\lt\{\langle\chi\rangle,h_i\rt\}  =
%        \frac{q_s F_{0s}}{T_{0s}} \pd{\langle\chi\rangle}{t} + C_{ii}\lt[h_i\rt].
%        \label{eqs:ief:ions}
%    \eeq
%

    \section{$k_\perp \rho_i \ll 1$: Kinetic Reduced MHD}
    \label{eqs:sec:krmhd}
    \subsection{Equations}

    Even though gyrokinetics is a reduced set of computationally tractable equations,
    solving them
    numerically can prove to be quite expensive \footnote{Though not impossible, there
    are numerous gyrokinetic codes used widely to study turbulence in magnetized plasmas.}. In this section, we present a simpler
    hybrid model which is derived by taking the $k_\perp \rho_i \ll 1$ limit of the
    gyrokinetic set of equations. This range of wavenumbers corresponds to the ``inertial range"
    of the turbulent cascade. In this limit, dynamics of Alfv\'{e}n waves decouples from
    that of the slow waves.  The Alfv\'{e}n waves satisfy reduced MHD, a system of
    equations that can be derived from MHD in the collisional limit, but are true even in the
    collisionless limit.
    
    Define stream and flux function $\Phi$ and $\Psi$ as,
    \beq
        \Phi = \frac{c}{B_0}\phi, \quad \Psi = - \frac{\Apar}{\sqrt{4 \pi m_in_{0i}}}
        \label{eqs:eq:PhiPsi}.
    \eeq
    The Alfv\'{e}nic turbulence then evolves according to the following (reduced MHD) equations:
    \beq
        \pd{\Psi}{t} = v_A \hat{\mb{b}}\cdot\nabla \Phi, \label{eqs:krmhd:epar}
    \eeq
    \beq
        \od{\nabla_\perp^2 \Phi}{t} = v_A \hat{\mb{b}}\cdot \nabla \nabla_\perp^2
        \Psi,\label{eqs:krmhd:vort}
    \eeq
    where $v_A = B_0/\sqrt{4 \pi m_i n_{0i}}$ is the Alfv\'{e}n velocity. This seemingly
    strange result of a collisional theory being valid at collisionless scales happens
    because the Alfv\'{e}nic part of the distribution function (the part that describes
    the $\EcrossB$ drift of the plasma and the magnetic fieldlines), is a 
    shifted Maxwellian with a mean perpendicular flow velocity $\mb{u}_\perp = \mb{u}_E =
    \hat{\mb{z}}\times\Phi$, the $\EcrossB$ velocity:     
    \beq
        f_i = \underbrace{\frac{n_{0i}}{\lt(\pi v_{thi}^2\rt)^{3/2}}\exp
        \lt[-\frac{\lt(\mb{v}_\perp-\mb{u}_E\rt)^2 + \vpar^2}{v_{thi}^2}
        \rt]}_{\text{Alfv\'{e}nic fluctuations}} +
        \underbrace{\frac{v_\perp^2}{v_{thi}^2}\frac{\dBpar}{B_0} F_{0i} +
        g}_{\text{Compressive fluctuations}}. \label{eqs:krmhd:deltafi}
    \eeq
    Since the Alfv\'{e}nic fluctuations do not alter the Maxwellian character of the
    distribution function, it is unsurprising that the equations satisfied by the
    Alfv\'{e}n waves are the same in the collisional and the collisionless limit.

    The compressive fluctuations still require a kinetic description in terms of the
    function $g$ (see \eqref{eqs:krmhd:deltafi}), $g$ turns out to be a
    kinetic passive
    scalar, which evolves according to a kinetic equation that involves the density 
    ($\delta n_e$) and the field strength ($\dBpar$) fluctuations, and is turbulently mixed 
    by the Alfv\'{e}nic turbulence. The density and field strength fluctuations in turn depend on
    $g$. The complete set of equations describing the compressive fluctuations are:
    \beq
        \od{g}{t} + \vpar \hat{\mb{b}} \cdot \nabla \lt[g + \lt(\frac{Z}{\tau}
        \frac{\delta n_e}{n_{0e}} +
        \frac{v_\perp^2}{v_{thi}^2}\frac{\dBpar}{B_0}\rt)F_{0i}\rt] = \lt\langle C_{ii}\lt[g
        + \frac{v_\perp^2}{v_{thi}^2} \frac{\dBpar}{B_0} F_{0i}\rt]\rt\rangle_{\mb{R}_i}
        ,\label{eqs:krmhd:gkin}
    \eeq
    \beq
        \frac{\delta n_e}{n_{0e}} = \lt[\frac{Z}{\tau} + 2\lt(1 +
        \frac{1}{\beta_i}\rt)\rt]^{-1}\frac{1}{n_{0i}}\int d^3 \mb{v}
        \lt[\frac{v_\perp^2}{v_{thi}^2} - 2 \lt(1 +
        \frac{1}{\beta_i}\rt)\rt]g,\label{eqs:krmhd:dne}
    \eeq
    \beq
        \frac{\dBpar}{B_0} = \lt[\frac{Z}{\tau} + 2\lt(1 +
        \frac{1}{\beta_i}\rt)\rt]^{-1}\frac{1}{n_{0i}}\int d^3 \mb{v}
        \lt[\frac{v_\perp^2}{v_{thi}^2} + \frac{Z}{\tau} \rt]g,\label{eqs:krmhd:dbpar}
    \eeq

    where 
    \beq
        \od{}{t} = \pd{}{t} + \lt\{\Phi,\ldots\rt\}, \quad \hat{\mb{b}}\cdot\nabla =
        \pd{}{z} + \frac{1}{v_A}\lt\{\Psi,\ldots\rt\},\label{eqs:krmhd:convder}
    \eeq
    and
    \beq
        Z = \frac{q_i}{q_e}, \quad \tau = \frac{T_{0i}}{T_{0e}}, \quad \beta_i =
        \frac{v_{thi}^2}{v_A^2} = \frac{8\pi n_{0i}T_{0i}}{B_0^2}.
    \eeq
    $C_{ii}$ is the gyro-averaged, linearized ion-ion collision operator, that acts on the non-Maxwellian part of
    the distribution function.

    \Eqsand{eqs:krmhd:epar}{eqs:krmhd:vort} can be rewritten in a more intuitive form in terms of the Elsasser potentials,
    \beq
        \xi^\pm = \Phi \pm \Psi.
    \eeq
    The Alfv\'{e}n waves then satisfy (see also \eqref{intro:krmhd:els} and the following
    discussion),
    \beq
    \pd{\nabla_\perp^2 \xi^\pm}{t} \mp v_A \pd{\nabla_\perp^2 \xi^\pm}{z} = - \frac{1}{2} \left[
    \{\xi^+, \nabla_\perp^2 \xi^- \} + \{\xi^-, \nabla_\perp^2 \xi^+ \} \mp \nabla_\perp^2
    \{\xi^+, \xi^-
    \} \right]. \label{eqs:krmhd:els} 
    \eeq
    In this form, the ``$+$" and ``$-$" potentials have physical interpretations---they are
    counter-propagating Alfv\'{e}n waves.
%    The left hand side of \eqref{eqs:krmhd:els} describes linear Alfv\'{e}n waves,
%    traveling up or down the fieldline. The nonlinear interaction between these waves is captured by
%    the right hand side. Observe that only counter-propagating Alfv\'{e}n waves interact
%    nonlinearly; these colliding Alfv\'{e}n waves give rise to the turbulent cascade by
%    transferring energy to smaller spatial scales.
%
    In the collisionless limit, \eqsdash{eqs:krmhd:gkin}{eqs:krmhd:dbpar} can be reduced
    to a simpler form as well. If the collision operator in \eqref{eqs:krmhd:gkin} is ignored,
    the $v_\perp$ dependence can be integrated over. Define function $g_n(\vpar)$ and
    $g_B(\vpar)$ such that,
    \beq
        \int d\vpar g_n = \frac{\delta n_e}{n_{0e}}, \quad \int d \vpar g_B =
        \frac{\dBpar}{B_0}. \label{eqs:eq:gnBdef}
    \eeq
    Then \eqref{eqs:krmhd:gkin} becomes two coupled equations,
    \bea
        \od{g_n}{t} + \vpar \hat{\mb{b}}\cdot \nabla g_n = -\lt[\frac{Z}{\tau} + 2\lt(1 +
        \frac{1}{\beta_i}\rt)\rt]^{-1} \vpar F_0(\vpar) \nonumber \\
        \times \hat{\mb{b}}\cdot \nabla
        \lt[\frac{Z}{\tau}\lt(1 + \frac{2}{\beta_i}\rt) \frac{\delta n_e}{n_{0e}}+
        \frac{2}{\beta_i}\frac{\dBpar}{B_0}\rt],\label{eqs:krmhd:gn}
    \eea
    \bea
        \od{g_B}{t} + \vpar \hat{\mb{b}}\cdot \nabla g_B = -\lt[\frac{Z}{\tau} + 2\lt(1 +
        \frac{1}{\beta_i}\rt)\rt]^{-1} \vpar F_0(\vpar) \nonumber \\
        \times \hat{\mb{b}}\cdot \nabla
        \lt[\frac{Z}{\tau}\lt(1 + \frac{Z}{\tau}\rt) \frac{\delta n_e}{n_{0e}}+
        \lt(2 + \frac{Z}{\tau}\rt)\frac{\dBpar}{B_0}\rt],\label{eqs:krmhd:gb}
    \eea
    where $F_0(\vpar) = \lt(1/\sqrt{\pi v_{thi}}\rt) \exp\lt(-\vpar^2/v_{thi}^2\rt)$ is a
    one dimensional Maxwellian in $\vpar$. Further define,
    \beq
        g^+ = \frac{1}{\sigma}\lt(1 + \frac{Z}{\tau}\rt)g_n + g_B, \quad g^- = g_n +
        \frac{1}{\sigma}\frac{\tau}{Z} g_B, \label{eqs:krmhd:gpmdef}
    \eeq
    where
    \beq
        \sigma = 1 + \frac{\tau}{Z} + \frac{1}{\beta_i} + \sqrt{\lt(1 +
        \frac{\tau}{Z}\rt)^2 + \frac{1}{\beta_i^2}}.
    \eeq
    Then \eqsand{eqs:krmhd:gn}{eqs:krmhd:gb} can be reduced to the following decoupled
    equations:
    \beq
    \od{g^\pm}{t} + \vpar \nabla_\parallel g^\pm  = \frac{\vpar F_0(\vpar)}{\Lambda^\pm}
    \hat{\mb{b}}\cdot\nabla \int d \vpar g^\pm, \label{eqs:krmhd:gpm}
    \eeq
    where 
    \beq
        \Lambda^\pm = -\frac{\tau}{Z} + \frac{1}{\beta_i} \pm \sqrt{\lt(1 +
        \frac{\tau}{Z}\rt)^2 + \frac{1}{\beta_i^2}}. \label{eqs:krmhd:Lambda}
    \eeq
    \Eqsand{eqs:krmhd:els}{eqs:krmhd:gpm} together constitute the KRMHD model.

    $g^+$ and $g^-$, as defined in \eqref{eqs:krmhd:gpmdef} do not have obvious physical
    meanings like the Elsasser variables for Alfv\'{e}nic turbulence. However,
    it is somewhat instructive to consider the large and small
    beta limits: for $\beta_i \gg 1$, $g^- \approx g_n$, while for $\beta_i \ll 1$, $g^+
    \approx g_B$. 

    \subsection{Conserved quantity}

    In the $k_\perp \rho_i \ll 1$ limit, the conserved quantity from
    \secref{eqs:sec:gk:consqty} splits into four parts which are all separately conserved:
    \beq
        W = W_{\text{AW}}^+ + W_{\text{AW}}^- + W_{\text{compr}}^+ + W_{\text{compr}}^-,
    \eeq
    where 
    \beq
        W_{\text{AW}}^\pm = \int d^3\mb{r} \frac{m_i n_{0i}}{2}
        \lt|\nabla_\perp\xi^\pm\rt|^2
    \eeq
    are free-energies of the right and left-going Alfv\'{e}nic fluctuations, and
    \beq
        W_{\text{compr}}^\pm = \int d^r\mb{r}\frac{n_{0i}T_{0i}}{2}\lt[\int d\vpar
        \frac{\lt(g^\pm\rt)^2}{F_0} - \frac{1}{\Lambda^\pm}\lt(\int d\vpar
        g^\pm\rt)^2\rt]
    \eeq
    are free-energies of the $+$ and $-$ components of the compressive fluctuations, as
    defined in the previous section.

